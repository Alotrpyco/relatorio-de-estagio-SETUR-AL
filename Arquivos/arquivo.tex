%carregamento dos pacotes para a codificação, fontes e personalização do relátorio
\documentclass[12pt,a4paper]{article}
\usepackage[utf8]{inputenc}
\usepackage[T1]{fontenc}
\usepackage[brazil]{babel}
\usepackage{setspace}
\usepackage{geometry}
\usepackage{parskip}
\usepackage{graphicx}
\usepackage{lipsum}

% Configurações de layout
\geometry{
	top=2.5cm,
	bottom=2.5cm,
	left=3cm,
	right=2cm
}
\setlength{\parindent}{1.25cm}
\onehalfspacing

% Início do relátorio
\begin{document}
	
	% capa
	\begin{titlepage}
		\centering
		\large
		\textbf{UNIVERSIDADE FEDERAL DE ALAGOAS} \\
		\textbf{PRÓ-REITORIA DE GRADUAÇÃO} \\
		\vfill
		
		\Huge \textbf{Relatório de Estágio Curricular} \\
		\vspace{0.5cm}
		\LARGE \textit{Superintendência Administrativa, Planejamento, Orçamento, Finanças e Contabilidade - SAPOFC.} \\
		\vfill
		
		\Large Sérgio Ricardo Vieira Torres Silva \\
		\vspace{1.5cm}
	\end{titlepage}
	
	% folha de identificação
	\thispagestyle{empty}
	\begin{center}
		\textbf{UNIVERSIDADE FEDERAL DE ALAGOAS} \\
		\textbf{PRÓ-REITORIA DE GRADUAÇÃO}\\
		\vspace{1cm}
		{Relatório de Estágio Curricular em}\\
		{Superintendência Administrativa de orçamento, finanças e contabilidade - SAPOFC.}
	\end{center}
	
	\vspace{1cm}
	
	\noindent \textbf{Dados do Estagiário} \\
	Nome: \underline{Sérgio Ricardo Vieira Torres Silva\hspace{0,2cm}} \\
	Registro Acadêmico: \underline{21211879\hspace{0,2cm}} \\
	Matrícula: \underline{40-0\hspace{0,2cm}} \\
	Curso e Período: \underline{Ciêcias Econômicas\hspace{0,2cm}} \\
	
	\vspace{0.5cm}
	\noindent \textbf{Dados do Local de Estágio} \\
	Empresa: \underline{Secretaria de Estado do Turismo\hspace{0,2cm}} \\
	Supervisor: \underline{Leonildo José Oliveira da Silva\hspace{0,2cm}} \\
	Nº de registro: \underline{21-3\hspace{0,2cm}} \\
	
	\vspace{0.5cm}
	\noindent \textbf{Período do Estágio} \\
	Início: \underline{\hspace{0,2cm}} \quad Término:\underline{\hspace{0,2cm}} \\
	Jornada de trabalho: \underline{30\hspace{0,2cm}} horas semanais \\
	Total de horas: \underline{\hspace{0,2cm}} horas em \underline{\hspace{0,2cm}} \\
	
	\vfill
	\begin{flushright}
		\centering Maceió - AL \\
		2025
	\end{flushright}
	
	\newpage
	
	% 1. introduçao
	\section{Introdução}
	
	\hspace*{1,5cm}  O estágio é realizado na Secretaria de Estado do Turismo - SETUR/AL, localizada na Rua Celso Piatti, Nº 280-372, Jaraguá, Maceió/AL, CEP 57022-210. A secretaria está situada no interior do Centro Cultural e de Exposições Ruth Cardoso. A SETUR/AL atende principalmente profissionais do setor turístico e órgãos governamentais, oferecendo suporte para o desenvolvimento e a gestão do turismo em Alagoas. Dessa forma, os serviços oferecidos incluem o planejamento estratégico, alinhado entre todos os superintendentes presentes na secretaria, juntamente com a secretária de turismo do estado e chefe de cabinete. Além disso, a administração financeira, o orçamento e a contabilidade também fazem parte desses serviços prestados. Em relação aos produtos elaborados, a secretaria inclui documentos técnicos, relatórios financeiros e planos orçamentários, que contribuem para a organização e o crescimento do setor turístico do estado de Alagoas.
	
	A equipe da Superintendência Administrativa de orçamento, finanças e contabilidade - SAPOFC é composta por diversos profissionais, incluindo Superintendente Administrativo e Financeiro; gerente executivo administrativo; assessores técnicos de aquisição, frota, patrimônio, almoxarifado e controle de consumo interno; gerente executivo orçamento, finanças e contabilidade; supervisora de planejamento; assessores técnicos de planejamento e orçamento; além de gerentes e assessores técnicos de tecnologia da informação. Cada funcionário desempenha funções específicas, como coordenação administrativa, controle financeiro, planejamento orçamentário, suporte em tecnologia e gestão de recursos materiais, podendo assim garantir o funcionamento da secretaria.

	
	A organização do espaço físico da Secretaria de Estado do Turismo de Alagoas - SETUR/AL, de acordo com o organograma, é dividida em três áreas: órgão colegiado, gestão finalística e gestão estratégica. Sendo assim, dentro da gestão estratégica, destaca-se a SAPOFC, que está subordinado à secretaria Executiva de Gestão Interna. Portanto, toda essa estrutura possibilita uma organização eficiente para o desenvolvimento das atividades realizadas na Secretaria de Estado do Turismo.

	
	% 2. atividades desenvolvidas
	\section{Atividades Desenvolvidas}
	
	\hspace*{1,5cm} As atividades desenvolvidas, além de auxiliar o superintendente no planejamento da secretaria e na previsão das receitas orçamentárias públicas geradas pelas atividades de locação do centro de convenções, envolvem a coleta diária de dados por meio do Sistema Integrado de Administração Financeira do Estado de Alagoas – SIAFE/AL. Esses dados são utilizados para a construção e atualização dos relatórios, por meio de painéis de indicadores criados no Excel e no Power BI, quando necessário. A construção desses relatórios, voltados ao acompanhamento dos empenhos, liquidações e pagamentos das despesas da secretaria, como passagens, diárias, adiantamentos e pagamentos para as empresas habilitadas em licitações, por meio de contratos e convênios, era necessária para a atualização das metas físicas e financeiras relativas às ações e aos programas do Plano Plurianual (PPA), que eram de minha competência. Além disso, com a supervisão do superintendente, pude participar do planejamento e da proposta da Lei Orçamentária Anual (PLOA). Dessa forma, projetamos o orçamento para o ano de 2026 e, assim, digitamos a proposta de receitas e despesas para as unidades orçamentárias 29032 e 29533, respectivamente.
	
	Por outro lado, além das reuniões de acompanhamento para discutir o progresso das atividades e identificar os instrumentos adotados para acompanhamento e avaliação, também foram realizados feedbacks regulares pelo superintendente sobre o desempenho e a 
	qualidade dos relatórios desenvolvidos. Em relação a orientação dada pelo superintendente, foram fornecidas instruções detalhadas sobre o uso do SIAFE/AL, 
	aconselhamento para aprimorar as habilidades com Power BI e Excel, além de apoio na resolução de problemas encontrados durante a criação dos relatórios e painéis de indicadores das despesas.  
	
	Portanto, os procedimentos desenvolvidos como prática de estágio incluem: coleta e análise dos dados financeiros por meio do SIAFE/AL, para obter insights através da elaboração de painéis; utilização do Sistema Eletrônico de Informações (SEI), quando necessário, para buscar confirmações de Notas de Empenho (NE’s), Notas de Liquidação (NL’s) e Ordens Bancárias (OB’s) em alguns processos, a fim de embasar os argumentos feitos no acompanhamento das ações.  

	
	% 3. Suporte Teórico para a Solução de Problemas
	\section{Suporte Teórico para a Solução de Problemas}
	\hspace*{1,5cm}
	
	% 4. conclusão
	\section{Conclusão}
	\hspace*{1,5cm} 
	
	% 5. anexos
	\section{Anexos}
	
	\newpage
	% 6. de acordo
	\vspace{2cm}
	\noindent \textbf{De Acordo:}
	
	\vfill

	\noindent
	\underline{\hspace{7cm}} \hfill \underline{\hspace{7cm}} \\[0.3cm]

	\begin{minipage}[t]{7cm}
		\centering
		\textbf{Carimbo e assinatura do\
		Supervisor}
	\end{minipage}
		\hfill
	\begin{minipage}[t]{7cm}
		\centering
		\textbf{Estagiário}
	\end{minipage}

	\end{document}
