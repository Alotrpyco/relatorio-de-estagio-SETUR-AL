%carregamento dos pacotes para a codificação, fontes e personalização do relátorio
\documentclass[12pt,a4paper]{article}
\usepackage[utf8]{inputenc}
\usepackage[T1]{fontenc}
\usepackage[brazil]{babel}
\usepackage{setspace}
\usepackage{geometry}
\usepackage{parskip}
\usepackage{graphicx}
\usepackage{lipsum}

% Configurações de layout
\geometry{
	top=2.5cm,
	bottom=2.5cm,
	left=3cm,
	right=2cm
}
\setlength{\parindent}{1.25cm}
\onehalfspacing

% Início do relátorio
\begin{document}
	
	% capa
	\begin{titlepage}
		\centering
		\large
		\textbf{UNIVERSIDADE FEDERAL DE ALAGOAS} \\
		\textbf{PRÓ-REITORIA DE GRADUAÇÃO} \\
		\vfill
		
		\Huge \textbf{Relatório de Estágio Curricular} \\
		\vspace{0.5cm}
		\LARGE \textit{Superintendência Administrativa, Planejamento, Orçamento, Finanças e Contabilidade - SAPOFC.} \\
		\vfill
		
		\Large Sérgio Ricardo Vieira Torres Silva \\
		\vspace{1.5cm}
	\end{titlepage}
	
	% folha de identificação
	\thispagestyle{empty}
	\begin{center}
		\textbf{UNIVERSIDADE FEDERAL DE ALAGOAS} 
		\textbf{PRÓ-REITORIA DE GRADUAÇÃO} 
		\vspace{1cm}
		\textbf{Relatório de Estágio Curricular em} \\
		\textit{Superintendência Administrativa de orçamento, finanças e contabilidade - SAPOFC.}
	\end{center}
	
	\vspace{1cm}
	
	\noindent \textbf{Dados do Estagiário} \\
	Nome: \underline{\hspace{12cm}} \\
	Registro Acadêmico: \underline{\hspace{10cm}} \\
	Curso e Período: \underline{\hspace{10cm}} \\
	
	\vspace{0.5cm}
	\noindent \textbf{Dados do Local de Estágio} \\
	Empresa: \underline{\hspace{13cm}} \\
	Supervisor: \underline{\hspace{12cm}} \\
	Nº de registro: \underline{\hspace{10cm}} \\
	
	\vspace{0.5cm}
	\noindent \textbf{Período do Estágio} \\
	Início: \underline{\hspace{4cm}} \quad Término: \underline{\hspace{4cm}} \\
	Jornada de trabalho: \underline{\hspace{5cm}} horas semanais \\
	Total de horas: \underline{\hspace{4cm}} horas em \underline{\hspace{6cm}} \\
	
	\vfill
	\begin{flushright}
		\centering Maceió - AL \\
		2025
	\end{flushright}
	
	\newpage
	
	% 1. introduçao
	\section{Introdução}
	
	\hspace*{1,5cm}
	
	% 2. atividades desenvolvidas
	\section{Atividades Desenvolvidas}
	


	\hspace*{1,5cm} As atividades desenvolvidas, além de auxiliar o superintendente no planejamento da secretaria e 
	na previsão das receitas orçamentárias públicas geradas pelas atividades de locação do centro de convenções, envolvem a coleta diária de dados por meio do Sistema Integrado de Administração Financeira do Estado de Alagoas – SIAFE/AL. Esses dados são utilizados para a construção e atualização dos relatórios, por meio de painéis de indicadores criados no Excel e no Power BI, quando necessário. A construção desses relatórios, voltados ao acompanhamento dos empenhos, liquidações e pagamentos das despesas da secretaria, como passagens, diárias, adiantamentos e pagamentos para as empresas habilitadas em licitações, por meio de contratos e convênios, era necessária para a atualização das metas físicas e financeiras relativas às ações e aos programas do Plano Plurianual (PPA), que eram de minha competência.  
	
	Por outro lado, além das reuniões de acompanhamento para discutir o progresso das atividades e identificar os instrumentos 
 	adotados para acompanhamento e avaliação, também foram realizados feedbacks regulares pelo superintendente sobre o desempenho e a 
 	qualidade dos relatórios desenvolvidos. Em relação a orientação dada pelo superintendente, foram fornecidas instruções detalhadas sobre o uso do SIAFE/AL, 
 	aconselhamento para aprimorar as habilidades com Power BI e Excel, além de apoio na resolução de problemas encontrados durante a criação dos relatórios e painéis de indicadores das despesas.  
 	
	Portanto, os procedimentos desenvolvidos como prática de estágio incluem: coleta e análise dos dados financeiros por meio do SIAFE/AL 
	para obter insights por meio da elaboração de painéis; utilização do Sistema Eletrônico de Informações (SEI) 
 	para buscar confirmações de NE’s, NL’s e OB’s em alguns processos, a fim de embasar os argumentos feitos no acompanhamento das ações.
 	Posto isto, faz-se necessária a apresentação do material colocado à disposição para estudo.


	% 3. Suporte Teórico para a Solução de Problemas
	\section{Suporte Teórico para a Solução de Problemas}
	\hspace*{1,5cm}
	
	% 4. conclusão
	\section{Conclusão}
	
	
	% 5. anexos
	\section{Anexos}
	
	\begin{itemize}
		\item Avaliação do supervisor (modelo MGE)
		\item Termo de compromisso assinado
	\end{itemize}
	
	% 6. de acordo
	\vspace{2cm}
	\noindent \textbf{De Acordo:}
	
	\vspace{2cm}
	\noindent \underline{\hspace{7cm}} \hfill \underline{\hspace{7cm}} \\
	\textbf{Supervisor (carimbo e assinatura)} \hfill \textbf{Estagiário}
	
\end{document}
