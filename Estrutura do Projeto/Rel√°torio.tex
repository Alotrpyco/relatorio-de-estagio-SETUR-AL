\documentclass[12pt,a4paper]{article}

% Pacotes necessários
\usepackage[utf8]{inputenc}
\usepackage[T1]{fontenc}
\usepackage[brazil]{babel}
\usepackage{setspace}
\usepackage{geometry}
\usepackage{parskip}
\usepackage{graphicx}
\usepackage{lipsum}

% Configurações de layout
\geometry{
	top=2.5cm,
	bottom=2.5cm,
	left=3cm,
	right=2cm
}
\setlength{\parindent}{1.25cm}
\onehalfspacing

% Início do documento
\begin{document}
	
	% CAPA
	\begin{titlepage}
		\centering
		\large
		\textbf{UNIVERSIDADE FEDERAL DE ALAGOAS} \\
		\textbf{PRÓ-REITORIA DE GRADUAÇÃO} \\
		\vfill
		
		\Huge \textbf{Relatório de Estágio Curricular em} \\
		\vspace{0.5cm}
		\LARGE \textit{(área ou setor do estágio)} \\
		\vfill
		
		\Large Sérgio Ricardo Vieira Torres Silva \\
		\vspace{1.5cm}
	\end{titlepage}
	
	% FOLHA DE IDENTIFICAÇÃO
	\thispagestyle{empty}
	\begin{center}
		\textbf{UNIVERSIDADE FEDERAL DE ALAGOAS} 
		\textbf{PRÓ-REITORIA DE GRADUAÇÃO} 
		\vspace{1cm}
		\textbf{Relatório de Estágio Curricular em} 
		\textit{(área ou setor do estágio)}
	\end{center}
	
	\vspace{1cm}
	
	\noindent \textbf{Dados do Estagiário} \\
	Nome: \underline{\hspace{12cm}} \\
	Registro Acadêmico: \underline{\hspace{10cm}} \\
	Curso e Período: \underline{\hspace{10cm}} \\
	
	\vspace{0.5cm}
	\noindent \textbf{Dados do Local de Estágio} \\
	Empresa: \underline{\hspace{13cm}} \\
	Supervisor: \underline{\hspace{12cm}} \\
	Nº de registro: \underline{\hspace{10cm}} \\
	
	\vspace{0.5cm}
	\noindent \textbf{Período do Estágio} \\
	Início: \underline{\hspace{4cm}} \quad Término: \underline{\hspace{4cm}} \\
	Jornada de trabalho: \underline{\hspace{5cm}} horas semanais \\
	Total de horas: \underline{\hspace{4cm}} horas em \underline{\hspace{6cm}} \\
	
	\vfill
	\begin{flushright}
		Maceió - AL \\
		2025
	\end{flushright}
	
	\newpage
	

	
	% 1. INTRODUÇÃO
	\section{Introdução}
	
	Descreva o local de estágio, o público atendido, os serviços oferecidos, produtos elaborados, tipos de materiais que compõem o acervo, organização do espaço físico, equipe e atividades exercidas pelos membros.
	
	% 2. ATIVIDADES DESENVOLVIDAS
	\section{Atividades Desenvolvidas}
	
	Descreva suas atividades, procedimentos práticos, instrumentos de avaliação, orientação recebida e bibliografia usada para o estágio.
	
	% 3. SUPORTE TEÓRICO PARA A SOLUÇÃO DE PROBLEMAS
	\section{Suporte Teórico para a Solução de Problemas}
	
	Apresente os livros, artigos e outras fontes utilizadas para lidar com problemas enfrentados durante o estágio. Referencie segundo as normas da ABNT (se necessário, com \texttt{abntex2cite}).
	
	% 4. CONCLUSÃO
	\section{Conclusão}
	
	Comente se o estágio foi satisfatório, como foi o contato com colegas da área e relacione o estágio com os conhecimentos adquiridos na graduação.
	
	% 5. ANEXOS
	\section{Anexos}
	
	\begin{itemize}
		\item Avaliação do supervisor (modelo MGE)
		\item Termo de compromisso assinado
	\end{itemize}
	
	% 6. DE ACORDO
	\vspace{2cm}
	\noindent \textbf{De Acordo:}
	
	\vspace{2cm}
	\noindent \underline{\hspace{7cm}} \hfill \underline{\hspace{7cm}} \\
	\textbf{Supervisor (carimbo e assinatura)} \hfill \textbf{Estagiário}
	
\end{document}
