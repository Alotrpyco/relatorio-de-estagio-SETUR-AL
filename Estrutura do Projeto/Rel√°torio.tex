\documentclass[12pt,a4paper]{article}

% Pacotes necessários
\usepackage[utf8]{inputenc}
\usepackage[T1]{fontenc}
\usepackage[brazil]{babel}
\usepackage{setspace}
\usepackage{geometry}
\usepackage{parskip}
\usepackage{graphicx}
\usepackage{lipsum}

% Configurações de layout
\geometry{
	top=2.5cm,
	bottom=2.5cm,
	left=3cm,
	right=2cm
}
\setlength{\parindent}{1.25cm}
\onehalfspacing

% Início do documento
\begin{document}
	
	% CAPA
	\begin{titlepage}
		\centering
		\large
		\textbf{UNIVERSIDADE FEDERAL DE ALAGOAS} \\
		\textbf{PRÓ-REITORIA DE GRADUAÇÃO} \\
		\vfill
		
		\Huge \textbf{Relatório de Estágio Curricular em} \\
		\vspace{0.5cm}
		\LARGE \textit{Superintendência Administrativa de orçamento, finanças e contabilidade - SAPOFC.} \\
		\vfill
		
		\Large Sérgio Ricardo Vieira Torres Silva \\
		\vspace{1.5cm}
	\end{titlepage}
	
	% FOLHA DE IDENTIFICAÇÃO
	\thispagestyle{empty}
	\begin{center}
		\textbf{UNIVERSIDADE FEDERAL DE ALAGOAS} 
		\textbf{PRÓ-REITORIA DE GRADUAÇÃO} 
		\vspace{1cm}
		\textbf{Relatório de Estágio Curricular em} \\
		\textit{Superintendência Administrativa de orçamento, finanças e contabilidade - SAPOFC.}
	\end{center}
	
	\vspace{1cm}
	
	\noindent \textbf{Dados do Estagiário} \\
	Nome: \underline{\hspace{12cm}} \\
	Registro Acadêmico: \underline{\hspace{10cm}} \\
	Curso e Período: \underline{\hspace{10cm}} \\
	
	\vspace{0.5cm}
	\noindent \textbf{Dados do Local de Estágio} \\
	Empresa: \underline{\hspace{13cm}} \\
	Supervisor: \underline{\hspace{12cm}} \\
	Nº de registro: \underline{\hspace{10cm}} \\
	
	\vspace{0.5cm}
	\noindent \textbf{Período do Estágio} \\
	Início: \underline{\hspace{4cm}} \quad Término: \underline{\hspace{4cm}} \\
	Jornada de trabalho: \underline{\hspace{5cm}} horas semanais \\
	Total de horas: \underline{\hspace{4cm}} horas em \underline{\hspace{6cm}} \\
	
	\vfill
	\begin{flushright}
		Maceió - AL \\
		2025
	\end{flushright}
	
	\newpage
	

	
	% 1. INTRODUÇÃO
	\section{Introdução}
	
\hspace*{1,5cm}O estágio é realizado na Secretaria de Estado do Turismo – SETUR/AL, localizada na
Rua Celso Piatti, No 280-372, Jaraguá, Maceió/AL, CEP 57022-210. A secretaria está situada
no interior do Centro Cultural e de Exposições Ruth Cardoso. Ademais, a estrutura da secretaria
se divide em: órgão colegiado, gestão finalística e gestão estratégica. Em relação à gestão
estratégica, o setor de Superintendência Administrativa, de Planejamento, Orçamento, Finanças
e Contabilidade (SAPOFC) está subordinado à Secretaria Executiva de Gestão Interna. Além
disso, esse setor é ocupado pelos seguintes cargos: gerente executivo administrativo, assessora
técnica de aquisição, assessor técnico de frota, patrimônio, almoxarifado e de controle de
consumo interno, gerente executivo de planejamento, orçamento, finanças e contabilidade,
supervisora de planejamento, orçamento, finanças e contabilidade, assessora técnica de
planejamento e orçamento, assessora técnica de finanças e contabilidade, gerente executiva de
valorização de pessoas, assessora técnica, assessora especial, gerente executivo de tecnologia
da informação e assessora técnica de infraestrutura de tecnologia da informação e suporte. Neste
setor – SAPOFC – onde todas as atividades listadas no plano de atividade executada por mim,
o público atendido são os próprios colaboradores de outros setores em busca de sanar as suas
dúvidas. Além disso, os serviços oferecidos estão diretamente ligados aos produtos elaborados.
Nesse sentido, os produtos elaborados são construções de painéis de indicadores das principais
despesas da secretaria, que servem para esclarecer dúvidas referente a empenho, liquidação e
pagamento.
	
	% 2. ATIVIDADES DESENVOLVIDAS
	\section{Atividades Desenvolvidas}
	
	Descreva suas atividades, procedimentos práticos, instrumentos de avaliação, orientação recebida e bibliografia usada para o estágio.
	
	% 3. SUPORTE TEÓRICO PARA A SOLUÇÃO DE PROBLEMAS
	\section{Suporte Teórico para a Solução de Problemas}
	
	Apresente os livros, artigos e outras fontes utilizadas para lidar com problemas enfrentados durante o estágio. Referencie segundo as normas da ABNT (se necessário, com \texttt{abntex2cite}).
	
	% 4. CONCLUSÃO
	\section{Conclusão}
	
	Comente se o estágio foi satisfatório, como foi o contato com colegas da área e relacione o estágio com os conhecimentos adquiridos na graduação.
	
	% 5. ANEXOS
	\section{Anexos}
	
	\begin{itemize}
		\item Avaliação do supervisor (modelo MGE)
		\item Termo de compromisso assinado
	\end{itemize}
	
	% 6. DE ACORDO
	\vspace{2cm}
	\noindent \textbf{De Acordo:}
	
	\vspace{2cm}
	\noindent \underline{\hspace{7cm}} \hfill \underline{\hspace{7cm}} \\
	\textbf{Supervisor (carimbo e assinatura)} \hfill \textbf{Estagiário}
	
\end{document}
